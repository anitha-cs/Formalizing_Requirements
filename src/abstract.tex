Formal methods have been enormously useful in verifying complex system requirements. However, their success depends on precisely formalizing {\em what} needs to be verified and thoroughly understanding {\em how} it is verified. While the advances in formal methods has given rise to sophisticated techniques and tools, there is a lack of awareness and methodological guidance in using these techniques effectively, that often makes their use difficult and the results of their application leading to overconfidence in the correctness of the fielded system in its intended environment.

In this paper, we report on using formal methods to verify a complex infusion pump system.  While the effort was successful and has led to end-to-end verification of a hierarchically composed software architecture, it was not without challenges that we believe are not adequately presented in the research literature. In our experience, we found that (a) precisely identifying the contextual information for requirements when formalizing them from traditionally structured requirements document is a non-trivial task, (b) some incorrect guidance exists on ``flowing down'' system requirements to lower levels of abstraction that could cause misinformed judgement about the correctness of the system, (c) inadequacy in understanding the technicalities of the formal tool can lead to ``proofs" based on faulty premises, and (d) inadequate mitigation of risks when using multiple analysis tools can result in misplaced confidence about the system. We then explain our approach to identify, mitigate and address such concerns.
