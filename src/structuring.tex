\subsection{Challenge 1: Requirements Formalization}

While model-based approaches helps verify requirements, the process of formalizing such requirements from the natural language documents is often a laborious process. One of the major reasons for this challenge comes from the fact that natural language depends heavily on context. To make a smooth transition from the natural language to the formal specification languages we need to systematically identify the contextual information for each requirement. When the contextual information is not sufficiently captured in the requirement and not easily identifiable from the requirements documents, formal tools such as model checkers return counter examples to help engineers discover the contextual information. However, the counter example directed context exploration is a time-consuming task especially for certain types of requirements.

When formalizing the requirements of GPCA, we found that identifying the context for two groups of requirements was challenging. The first set of requirements were those that describe the behaviour of the system under under normal working conditions. For example, one of the requirements states\footnote{\scriptsize{We intensionally simplified this requirement such that it illustrates the problem. However, the original requirement had more conditions associated with it.}} that,

\begin{quotation}
\emph{``When the patient requests a bolus, the system shall deliver an the drug at flow rate equal to $patient\_flow\_rate$ ''}
\end{quotation}

When we tried to formalize the requirement as is and verify it, the model checker repeatedly returned counter examples. A careful examination of the counter examples revealed that there were certain conditions in the system that prevented the patient bolus infusion to occur and hence the requirement was not satisfied in that context. These conditions were actually safety features of the system to prevent hazards that were documented as ``alarm requirements" in another section in the requirements document.  Unfortunately, traceability between such requirements was neither available nor establishing and maintaining then was practically straightforward, due to the numerous requirements and the orthogonality/exclusivity between the behaviours the requirements capture.

The second group of requirements that we had trouble formalizing were mutually exclusive requirements with a certain inherent priority among them. In the GPCA, the alarming requirements capture the system's responses to exceptional conditions. There were 18 exceptional conditions identified for the GPCA, each with its own set of desired system responses depending upon the severity level of that condition. For example, if the drug reservoir is empty, one of the highest severity condition, the system shall raise audio alarm, display error message and stop infusion. On the other hand, when the system is idle for a long time, a low severity level condition, the system only displays an appropriate message. One of the problems we encountered was formalizing the requirement in such a way that verifies their effect independently. For instance, to verify a lower priority condition requirement we had to systematically ensure all the higher priority ones did not occur. The problem was that there was no explicit priority specified in the requirements document and we had to infer the priority based on the system responses. Unfortunately, there was no guidance or patterns to help systematically organize and formalize such requirements for verification. On the contrary, the GPCA model that implements the requirement had a specific prioritization mechanism (that we believe is a design decision of the developer implementing the requirements). Formalizing and verifying each alarm condition requirements independently without including the design decisions of the model was a big challenge. We realized that root cause of this problem is the undisciplined organization of the requirements statements within the document.
