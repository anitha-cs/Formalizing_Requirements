\subsection{Challenge 1: Requirements Formalization}

While model-based approaches helps verify requirements, the process of formalizing requirements from the natural language documents is often a error-prone and laborious process. Although, natural language (NL) has is the practical choice for capturing requirements when it comes to rigorously verifying them, formal notation is the solution. When the systems are complex - which is mostly the case - the process of formalizing requirements becomes a non-trivial activity. The challenge arises from both the ambiguity and implicit contextual knowledge in the NL statements. While the process of formalization implicity takes care of the ambiguity, precisely identifying he context of the requirement is a painful process. By contextual nature of requirements, we mean the specific state of the system in which the requirements need to hold; in formal terms it is the antecedent or precondition in a formal statement. Even the well known specification patterns focus on recapturing the NL statements to formal notations and does not help address the challenge of identifying the context of the requirements. To partially address this concern, in domains such as safety critical systems, the formalized requirements are verified with respect to a model of the system. When the context is not sufficiently captured in the formalized requirement, tools such as model checkers return counter examples to help engineers discover the contextual information. However, the counter example directed context exploration is a very time-consuming task, especially for certain types of requirements.

When formalizing the requirements of GPCA, we found that identifying the context for two groups of requirements was challenging and tiring. The first set of requirements were those that describe the behaviour of the system under under normal working conditions (those that were not safety requirements). For example, one of the requirements states\footnote{\scriptsize{We intensionally simplified this requirement such that it illustrates the problem. However, the original requirement had more conditions associated with it.}} that,

\begin{quotation}
\emph{``When the patient requests a bolus, the system shall deliver an the drug at flow rate equal to $patient\_flow\_rate$ ''}
\end{quotation}

When we tried to formalize the requirement as is and verify it, the model checker repeatedly returned counter examples. A careful examination of the counter examples revealed that there were certain conditions in the system that prevented the patient bolus infusion to occur and hence the requirement was not satisfied in that context. These conditions were actually safety features of the system to prevent hazards that were documented as ``alarm requirements" in another section in the requirements document.  Similarly, consider another requirement that describes how basal infusion that is typically scheduled to infuse a certain quantity of drug over a specific period of time should operate. To verify this requirement, we needed to identify all the system conditions that can override the basal infusion. This included bolus infusions that have higher priority, hazards conditions that prevent infusion, clinician's manual pause or cancel that can temporarily suspend or abruptly stop infusion, etc. Again, these system conditions were present in the document in the form of other requirements, but organized in a manner that was unsystematic for formalization. Unfortunately, traceability between such requirements was neither available nor easily establishable and maintainable in practice, due to the numerousness of the requirements and the complexity in understanding the dependency between the
behaviours they capture.

The second group of requirements that was troublesome were the mutually exclusive requirements with a certain inherent priority among them. In the GPCA, the alarming requirements capture the system's responses to exceptional conditions. There were 18 exceptional conditions identified for the GPCA, each with its own set of desired system responses depending upon the severity level of that condition. For example, if the drug reservoir is empty (a high priority) the system shall raise audio alarm, display error message and stop infusion. On the other hand, when the system is idle for a long time (low priority), the system shall only display an appropriate message. One of the problems we encountered was trying to formalize it in such a way that their independent effect is verifiable. For instance, to verify a requirement with a lower priority condition we had to systematically capture the absence of all the higher priority ones. The challenge was systematically identifying the priority when it was not explicit specified. Unfortunately, there was no guidance or patterns to help us systematically organize and formalize such requirements for verification. On the contrary, the GPCA model had a specific prioritization mechanism (that we believe is a design decision of the developer implementing the requirements). Formalizing and verifying each alarm condition requirements independently without including the design decisions of the model was a big challenge. We realized that root cause of this problem is the undisciplined organization of the requirements statements within the document.
