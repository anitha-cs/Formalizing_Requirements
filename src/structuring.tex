\subsection{Challenge 1: Requirements Formalization}

While model-based approaches help to verify requirements, the process of formalizing such requirements from the natural language documents is often a laborious process. Although, natural language (NL) is the practical choice for capturing requirements, formal methods are necessary to rigorously verify them. For complex systems, the process of formalizing requirements becomes a non-trivial activity. The challenge arises from both the ambiguity and implicit contextual knowledge in the NL statements. While the process of formalization implicity takes care of the ambiguity, precisely identifying the context of the requirement is a painful process. By contextual nature of requirements, we mean the specific state of the system in which the requirements need to hold; in formal terms it is the antecedent or precondition in a formal statement. While well known specification patterns focus on recapturing the NL statements to formal notations, they do not help to address the challenge of identifying the context of the requirements. To partially address this concern in domains such as safety critical systems, the formalized requirements are verified with respect to a model of the system. When the context is not sufficiently captured in the formalized requirement, tools such as model checkers return counterexamples to help engineers discover the needed contextual information. However, this counterexample directed context exploration is a very time-consuming task, especially for certain types of requirements.

When formalizing the requirements of the GPCA, we found two groups of requirements whose context was particularly challenging to identify. The first set of requirements were those that describe the behaviour of the system under under normal working conditions. For example, one requirement states\footnote{\scriptsize{We intensionally simplified this requirement such that it illustrates the problem. However, the original requirement had more conditions associated with it.}} that,

\begin{quotation}
\emph{``When the patient requests a bolus, the system shall deliver an the drug at flow rate equal to $patient\_flow\_rate$ ''}
\end{quotation}

When we initially tried to formalize and verify the requirement, the model checker repeatedly returned counter examples. A careful examination of the counter examples revealed that there were certain conditions in the system that prevented the patient bolus infusion from occurring and hence the requirement was not satisfied in that context. These conditions were actually safety features of the system to prevent hazards that were documented as ``alarm requirements" in another section in the requirements document.  Unfortunately, traceability between such requirements was neither available nor establishing and maintaining then was practically straightforward, due to the numerous requirements and the orthogonality/exclusivity between the behaviours the requirements capture.

The second group of requirements that were difficult to formalize were those that were mutually exclusive, but had a certain inherent priority among them. In the GPCA, the alarm requirements capture the system's responses to exceptional conditions. There were 18 exceptional conditions identified for the GPCA, each with its own set of desired system responses depending upon the severity level of that condition. For example, the system responds to a high severity condition such as an empty drug reservoir by raising an audio alarm, displaying error message, and stopping infusion. On the other hand, when the system has been idle for a long time and a low severity level alarm is triggered, the system only displays an appropriate message. One of the problems we encountered was formalizing the requirement in such a way that verifies their effect independently. For instance, to verify a lower priority condition requirement we had to systematically ensure all the higher priority ones did not occur. The problem was that there was no explicit priority specified in the requirements document and we had to infer the priority based on the system responses. Unfortunately, there was no guidance or patterns to help systematically organize and formalize such requirements for verification. On the contrary, the GPCA model that implements the requirement had a specific prioritization mechanism (that we believe is a design decision of the developer implementing the requirements). Formalizing and verifying each alarm condition requirements independently without including the design decisions of the model was a big challenge. We realized that root cause of this problem is the undisciplined organization of the requirements statements within the document.
