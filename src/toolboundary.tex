\subsection{Challenge 4: Matching Tool Boundaries}

While it is ideal to use a tool/technique to verify the entire system's requirements, in practice it is inevitable to avoid the use of multiple tools for scalability concerns and different verification needs. One of the well known problems associated with using multiple tools is matching the sematic differences between them. In an earlier effort, in which we semi-formally mapped abstraction differences between infusion pump requirements formalized and captured as timed automata and AGREE notation, we identified semantic errors in the the mapping, in particular differences in the way time is handled in the notations. We fixed those issues by defining a formal semantics~\cite{whalen2015hierarchical} to rigorously map the models and their requirements. However, we found that even when tools with ``same" semantics were used rigorously maintaining the relationship between them in the long run is challenge.

In GPCA, we used a combination of two tools to cope with scalability of verification. We used AGREE to model the system architecture and verify the decomposition of system requirements into component requirements. Subsequently, we also developed detailed behavioral models for each component in Simulink and verified if the component requirements verified in AGREE are indeed satisfied by the respective component's behavioral model using Simulink Design Verifier (SDV). The advantage of this tool combination in addition to being scalable is that the semantics of the notation used in AGREE and SDV (Embedded Matlab) to specify requirements are comparable. Hence, we presumed that the translation of requirements between AGREE and Simulink notations can be done in a error free manner.

While a majority of the requirements were correctly recaptured between the notations some requirements were incorrectly recaptured that, unfortunately, was not easy to detect. We identified two such issues - transcription errors and requirement management issues. The transcription errors occurred in the process of manually recapturing requirements. When recapturing some requirements we unknowingly changed the operators. In GPCA, we translated the following requirements that were proved in Simulink into AGREE as :

\scriptsize{
\begin{verbatim}
Properties in Embedded Matlab:
-----------------------------
sldv.prove(
    implies(SystemOn and AirInLine),(FlowRate <= MedFlow))
sldv.prove(
    implies(SystemOn and EmptyReservoir),(FlowRate <= LowFlow))

Property in AGREE:
----------------------------
guarantee "Property1":
    (SystemOn && AirInLine) => (FlowRate $=$ MedFlow)
guarantee "Property2":
    (SystemOn && EmptyReservoir) => (FlowRate $=$ LowFlow)
\end{verbatim}
}
\normalsize{}

It was not possible to dismiss this error as a mere typo, since it demonstrates the potential for occurrence of such issues. Unfortunately, this error was not visible until we checked for feasibility using a recent enhancement of AGREE's tool called \emph{realizability}. Intuitively, realizability checks if there is exists an output of the system for every input that satisfies the requirements. In the above case, the realizability check identified a counterexample in which both AirInLine and EmptyReservoir\footnote{AirInLine indicates a hazard in which air bubble(s) are detected in the infusion tubing and EmptyReservoir is a hazard that indicates that the drug reservoir does not have sufficient drug to infuse}. While this feature was not intended to identify the recapture errors, coincidentally we were able to identify this issue. After the found this errors, on manual inspection we identified 6 such transcription issues and corrected them.

Another issue, that we believe is more common, is maintaining the synchrony between the requirements in both tools. After we proved requirements in AGREE, when we recaptured it for verification using SDV, at times we had to slightly change the way requirements were formalized, such that it is verifiable in the behavioral Simulink. However, such changes were sometimes missed to be changed and reverified in AGREE - another human error that caused mismatch between the requirements. Although we were diligent in recapturing the requirements between the tools, human errors and negligence was unavoidable. For the GPCA, we did additional manual inspections to verify the synchrony. However, in our opinion, a lack of rigours process and/or automation to translate and check for synchrony is the root cause of this problem. %to avoid such issues. Infact, our recommendation to engineers performing such tasks is to ensure there is either a strict process or automation to avoid such issues.

\subsubsection {Solution: Automated Translators and Manual Inspections}

To address the transcription errors for GPCA, we first strengthened the process of verifying the translation using code inspection by a different developers. This inspection was effective in identifying errors, but was time consuming. In the long run, to maintain the synchrony between the requirements in both the tools was challenging. This challenge became the motivation for another team who are currently developing an automated translator that recaptures properties from AGREE tool notation to Embedded Matlab notation and vice-versa. While we are not actively involved in the development of the automation, we are currently verifying and validating their work using the GPCA's manual translation. While code inspections were effective, automated translators significantly reduced the overall time in the overall process. 